\documentclass[a4paper,UTF8]{ctexart}
%\usepackage{ctex}使用该句会报redefine的bug
\usepackage[margin=1.25in]{geometry}
\usepackage{color}
\usepackage{graphicx}
\usepackage{amssymb}
\usepackage{amsmath}
\usepackage{amsthm}
%\usepackage[thmmarks, amsmath, thref]{ntheorem}
\theoremstyle{definition}
\newtheorem*{solution}{Solution}
\newtheorem*{prove}{Proof}
\usepackage{multirow}
\usepackage{url}
\usepackage{enumerate}
\usepackage{enumitem}
\usepackage{algorithm}
\usepackage{algorithmic}




%code
%\usepackage[utf8]{inputenc}
 
\usepackage{listings}
\usepackage{color}
 
\definecolor{codegreen}{rgb}{0,0.6,0}
\definecolor{codegray}{rgb}{0.5,0.5,0.5}
\definecolor{codepurple}{rgb}{0.58,0,0.82}
\definecolor{backcolour}{rgb}{0.95,0.95,0.92}
 
\lstdefinestyle{mystyle}{
    backgroundcolor=\color{backcolour},   
    commentstyle=\color{codegreen},
    keywordstyle=\color{magenta},
    numberstyle=\tiny\color{codegray},
    stringstyle=\color{codepurple},
    basicstyle=\footnotesize,
    breakatwhitespace=false,         
    breaklines=true,                 
    captionpos=b,                    
    keepspaces=true,                 
    numbers=left,                    
    numbersep=5pt,                  
    showspaces=false,                
    showstringspaces=false,
    showtabs=false,                  
    tabsize=2
}
 
\lstset{style=mystyle}
%code











\renewcommand{\algorithmicrequire}{\textbf{Input:}}
\renewcommand{\algorithmicensure}{\textbf{Procedure:}}
\renewcommand\refname{参考文献}

%--

%--
\begin{document}
\title{丰富的特征结构,用于精确的对象检测和语义分割}
\author{Ross Girshick Jeff Donahue Trevor Darrell Jitendra Malik UC Berkeley}
\maketitle
\begin{abstract}
对象检测性能,在标准(canonical) PASCAL VOC 数据集上测量,在最近几年一直稳定(plateau).最优性能的方法是复杂集成系统(complex ensemble system)典型地联合(combine)多个(multiple)低层次的图像特征和高层次的内容。在这篇paper中,我们提出一个简单并且可扩展的检测算法,提升了平均精度mAP(mean average precision)30个百分点,相对于之前在VOC 2012 中最好的结果-获得 53.3\%的mAP.我们的方法联合两个关键的见解(key insights):(1)人们可以将高容量(high-capacity)卷积神经网络(CNN)应用于自下而上的区域候选以便定位和分割对象.以及(2) 当有标记训练数据稀缺(scarce)时,一个辅助(auxiliary)任务的监督预训练,紧跟着一个特定领域的微调(domain-specific fine-tuning),这种训练方式,产生了(yield)一个显著的性能提升(signficant performance boost),因为我们结合了区域候选和CNN,所以我们叫我们的方法R-CNN:结合CNN特征的区域.整个系统的源代码提供地址是

\section{介绍}
特征很重要(feature matter).上个世纪在不同视觉识别任务上的进展颇(considerably)基于SIFT和HOG的使用。但是如果我们看在经典(canonical)视觉识别任务--PASCAL VOC物体检测,上的性能,被广泛承认的是,在2010-2012间进展缓慢,,只有通过建立集成系统和使用成功方法的小变体(minor variant)获得的小收益(small gains)
SIFT 和HOG 是块定向直方图(blockwise orientation histogram),一种我们可以与V1层中的细胞粗糙(roughly)地联系起来的特征表示.V1层的细胞是灵长类(primate)视觉通路(visual pathway)的第一皮质区域(first cortical area).但是我们也知道识别
\end{abstract}
\end{document}